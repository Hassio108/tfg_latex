\chapter{Ejemplo de capítulo}

[Esto es un ejemplo de capítulo]

Lo que sigue es un lorem ipsum como ejemplo de lo que sería un capítulo. Incluimos esta cita bibliográfica \cite{recomendaciones}.

\lipsum

\section{Ejemplo de inclusión de código fuente}

A continuación se muestra el mismo listado de código usando los listings y minted:


\subsection{Código con listings}

\begin{lstlisting}[language=c]
#include <stdio.h>
// A simple Hello World
int main(){
  printf("Hello World!\n");
  return 0;
}
\end{lstlisting}

\subsection{Código con minted}

\begin{minted}{c}
#include <stdio.h>
// A simple Hello World
int main(){
  printf("Hello World!\n");
  return 0;
}
\end{minted}

\section{Ejemplo de cita bibliográfica}

Esta es la cita bibliográfica de un libro \cite{ec}.

%%% Local Variables: 
%%% mode: latex
%%% TeX-master: "tfg_main"
%%% TeX-PDF-mode: t
%%% ispell-local-dictionary: "castellano"
%%% End: 
