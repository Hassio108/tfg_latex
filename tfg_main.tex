\documentclass[twoside,12pt]{book}
\usepackage[utf8]{inputenc}
\usepackage[T1]{fontenc}
\usepackage[english,spanish]{babel}
\usepackage[left=3cm, right=3cm, top=3.5cm, bottom=3.5cm]{geometry} % Márgenes recomendados
\usepackage{times}
\usepackage[scaled=0.75]{beramono}
\usepackage{portada}
\usepackage[style=ieee, backend=biber]{biblatex} % Bibliografía en formato IEEE
\setlength{\parskip}{1.5ex} %% Separamos los párrafos un poquito más

%%%%%%%%%%%%%%%%%%%%%%%%%%%%%%%%%%%%%%%%%%%%%%%%%%%%%%%%%%%%%%%%%%%%%%%%
%% Datos del director y del autor
\Director{Nombre del Director}
%\Lugar{Bilbao} % Por omisión: Madrid
%\Grado{Graduado en Matemáticas e Informática} % Por omisión: Graduado en Ingeniería Informática
%\Trabajo{TRABAJO FIN DE MÁSTER} % Por omisión: TRABAJO FIN DE GRADO
\author{Nombre del Alumno}
\date{Enero de 2017}
\title{Título del trabajo}
\addbibresource{biblio.bib} %% ficheros con la bibliografía

%%%%%%%%%%%%%%%%%%%%%%%%%%%%%%%%%%%%%%%%%%%%%%%%%%%%%%%%%%%%%%%%%%%%%%%%
%% Algunos paquetes interesantes de los que depende la plantilla.
\usepackage{hyperref}
\usepackage{fancyhdr}

%%%%%%%%%%%%%%%%%%%%%%%%%%%%%%%%%%%%%%%%%%%%%%%%%%%%%%%%%%%%%%%%%%%%%%%%
%% Otros paquetes
\usepackage{lipsum}
\usepackage[newfloat]{minted}
\usepackage{caption}
\newenvironment{code}{\captionsetup{type=listing}}{}
\SetupFloatingEnvironment{listing}{name=Listado}
\usepackage{graphicx}

%%%%%%%%%%%%%%%%%%%%%%%%%%%%%%%%%%%%%%%%%%%%%%%%%%%%%%%%%%%%%%%%%%%%%%%%
%% Tipos de letra en cabeceras de secciones

\newcommand{\dedicafont}{\usefont{T1}{phv}{m}{n}\fontsize{14}{17.5}\selectfont}
\newcommand*{\copyrightfont}{\usefont{T1}{phv}{m}{n}\fontsize{6}{7.5}\selectfont}
\newcommand{\sectionalfont}{\usefont{T1}{phv}{b}{n}}
% {64}{80} {56}{70} {48}{60} {32}{40} {24}{30} {18}{22.5} {16}{20} {14}{17.5} {12}{15}
\newcommand{\partnumberfont}{\sectionalfont\fontsize{64}{80}\selectfont}
\newcommand{\parttitlefont}{\sectionalfont\fontsize{48}{60}\selectfont}
\newcommand{\chapternumberfont}{\sectionalfont\fontsize{64}{80}\selectfont}
\newcommand{\chaptertitlefont}{\sectionalfont\fontsize{32}{40}\selectfont}
\newcommand{\sectiontitlefont}{\sectionalfont\fontsize{18}{22.5}\selectfont}
\newcommand{\subsectiontitlefont}{\sectionalfont\fontsize{16}{20}\selectfont}
\newcommand{\subsubsectiontitlefont}{\sectionalfont\fontsize{14}{17.5}\selectfont}
\newcommand{\headertitlefont}{\sectionalfont\fontsize{12}{15}\selectfont}

%%%%%%%%%%%%%%%%%%%%%%%%%%%%%%%%%%%%%%%%%%%%%%%%%%%%%%%%%%%%%%%%%%%%%%%%
%% Formateo de capítulos y secciones
\usepackage{titlesec}
\titleformat{\chapter} % command
[display] % shape
{} % format
{\chapternumberfont\flushright\thechapter} % label
{0pt} % sep
{\chaptertitlefont\flushright} % before
[] % after

\titleformat{\section} % command
[block] % shape
{} % format
{\sectiontitlefont\thesection} % label
{1.5ex} % sep
{\sectiontitlefont} % before
[] % after


\titleformat{\subsection} % command
[block] % shape
{} % format
{\subsectiontitlefont\thesubsection} % label
{1.5ex} % sep
{\subsectiontitlefont} % before
[] % after


\titleformat{\subsubsection} % command
[block] % shape
{} % format
{\subsubsectiontitlefont\thesubsubsection} % label
{1.5ex} % sep
{\subsubsectiontitlefont} % before
[] % after

%%%%%%%%%%%%%%%%%%%%%%%%%%%%%%%%%%%%%%%%%%%%%%%%%%%%%%%%%%%%%%%%%%%%%%%%
%% Empieza el documento
\begin{document}

\pagestyle{empty}

%%%%%%%%%%%%%%%%%%%%%%%%%%%%%%%%%%%%%%%%%%%%%%%%%%%%%%%%%%%%%%%%%%%%%%%% 
%% Portada
\maketitle
\null
\cleardoublepage

%%%%%%%%%%%%%%%%%%%%%%%%%%%%%%%%%%%%%%%%%%%%%%%%%%%%%%%%%%%%%%%%%%%%%%%%
%% Dedicatoria
\hfill{\itshape [Aquí va la dedicatoria]}

%%% Local Variables: 
%%% mode: latex
%%% TeX-master: "tfg_main"
%%% TeX-PDF-mode: t
%%% ispell-local-dictionary: "castellano"
%%% End: 

\cleardoublepage

\pagenumbering{roman} % La numeración debe ser romana hasta el primer capítulo
\pagestyle{plain}
%%%%%%%%%%%%%%%%%%%%%%%%%%%%%%%%%%%%%%%%%%%%%%%%%%%%%%%%%%%%%%%%%%%%%%%%
%% Índice de contenidos
\tableofcontents
\cleardoublepage

%%%%%%%%%%%%%%%%%%%%%%%%%%%%%%%%%%%%%%%%%%%%%%%%%%%%%%%%%%%%%%%%%%%%%%%%
%% Otros índices
\listoffigures
\cleardoublepage

\listoftables
\cleardoublepage

%%%%%%%%%%%%%%%%%%%%%%%%%%%%%%%%%%%%%%%%%%%%%%%%%%%%%%%%%%%%%%%%%%%%%%%%
%% Resumen en castellano
\chapter*{Resumen}
\addcontentsline{toc}{chapter}{Resumen}

Este es el resumen del TFG.

\lipsum[1]

%%% Local Variables: 
%%% mode: latex
%%% TeX-master: "tfg_main"
%%% TeX-PDF-mode: t
%%% ispell-local-dictionary: "castelano"
%%% End: 

\cleardoublepage

%%%%%%%%%%%%%%%%%%%%%%%%%%%%%%%%%%%%%%%%%%%%%%%%%%%%%%%%%%%%%%%%%%%%%%%%
%% Abstract in English
\begin{otherlanguage}{english}

\chapter*{Abstract}
\addcontentsline{toc}{chapter}{Abstract}

[This is the abstract in English.]

\lipsum[1]

\end{otherlanguage}

%%% Local Variables: 
%%% mode: latex
%%% TeX-master: "tfg_main"
%%% TeX-PDF-mode: t
%%% ispell-local-dictionary: "english"
%%% End: 

\cleardoublepage

\pagenumbering{arabic} % Iniciamos la numeración árabe en el primer capítulo
\pagestyle{fancy}
% Remembering the chapter title:
\renewcommand{\chaptermark}[1]{\markboth{\headertitlefont\thechapter\quad#1}{}}
% Remembering section number and title:
\renewcommand{\sectionmark}[1]{\markright{\headertitlefont\thesection\quad#1}}
% Clear all header fields
\fancyhead{}
\fancyhead[RE]{}{}
\fancyhead[LE]{\leftmark}
\fancyhead[RO]{\rightmark}
% Clear all footer fields
\fancyfoot{}
\fancyfoot[C]{\thepage}
\renewcommand{\headrulewidth}{0pt}
\renewcommand{\footrulewidth}{0pt}

%%%%%%%%%%%%%%%%%%%%%%%%%%%%%%%%%%%%%%%%%%%%%%%%%%%%%%%%%%%%%%%%%%%%%%%%
%% Capítulos
\chapter{Ejemplo de capítulo}

[Esto es un ejemplo de capítulo]

Lo que sigue es un lorem ipsum como ejemplo de lo que sería un capítulo. Incluimos esta cita bibliográfica \cite{recomendaciones}

\lipsum


%%% Local Variables: 
%%% mode: latex
%%% TeX-master: "tfg_main"
%%% TeX-PDF-mode: t
%%% ispell-local-dictionary: "castellano"
%%% End: 

\cleardoublepage

%%%%%%%%%%%%%%%%%%%%%%%%%%%%%%%%%%%%%%%%%%%%%%%%%%%%%%%%%%%%%%%%%%%%%%%%
%% Bibliografía
\printbibliography[heading=bibintoc]

\end{document}

%%% Local Variables: 
%%% mode: latex
%%% TeX-master: t
%%% TeX-PDF-mode: t
%%% ispell-local-dictionary: "castellano"
%%% End: 
